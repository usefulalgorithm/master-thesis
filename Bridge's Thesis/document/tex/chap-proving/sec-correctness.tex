
\section{Correctness}\label{sec:correctness}

We are ready to sketch the proof of Theorem~\ref{theorem:soundness}.
In the case that the program \prog{P} passed our analysis,
and Algorithm~\ref{algorithm:overview} returns
$\mathit{Pass} (\Pi (\underline{G}^{\fun{main}}_k, \TT))$ and
$S[\bullet]$ on the input control flow graph $G^\fun{main} = \langle V, E,
\cmd{cmd}^\fun{main}, \overline{\mathtt{u}}^\fun{main},
\overline{\mathtt{r}}^\fun{main},s,e \rangle$,
the assertions in original program \prog{P} should hold.

Let $\underline{G}^{\fun{main}}_k = \langle \underline{V}_k, \underline{E}_k,
\underline{\cmd{cmd}}_k^\fun{main} , \overline{\mathtt{u}}^\fun{main},
\overline{\mathtt{r}}^\fun{main},s,e \rangle$ and
$\Pi (\underline{G}^{\fun{main}}_k, \TT)
= \{ \underline{I}_{\ell}: \ell \in \underline{V}_k \}$.
By the definition of inductive invariants, we have
$\assert{\underline{I}_{\ell}}\ \underline{\cmd{cmd}}_k^\fun{main} (\ell, \ell')\ \assert{\underline{I}_{\ell'}}$
for every $(\ell, \ell') \in  \underline{E}_k$. Moreover, $V \subseteq  \underline{V}_k$ since
$\underline{G}^{\fun{main}}_k$ is obtained by unwinding $G^\fun{main}$. Define 
$\Gamma (G^\fun{main}, \TT) = \{ \underline{I}_{\ell} \in \Pi (\underline{G}^{\fun{main}}_k,
\TT) : \ell \in V \}$. We claim $\Gamma (G^\fun{main}, \TT)$
is in fact an inductive invariant for $G^\fun{main}$. 

Let $\hat{E} = \{ (\ell, \ell') \in E : \cmd{cmd}^\fun{main} (\ell, \ell') =
\overline{\mathtt{x}} := \fun{f} (\overline{p}) \}$. We have
$\cmd{cmd}^\fun{main} (\ell, \ell') = \underline{\cmd{cmd}}_k^\fun{main} (\ell, \ell')$ for every
$(\ell, \ell') \in E \setminus \hat{E}$. Thus $\assert{\underline{I}_{\ell}}\
\cmd{cmd}^\fun{main} (\ell, \ell')\ \assert{\underline{I}_{\ell'}}$ for every $(\ell,
\ell') \in E \setminus \hat{E}$ by the definition of $\Gamma (G,
\TT)$ and the inductiveness of $\Pi (\underline{G}^{\fun{main}}_k,
\TT)$. It suffices to show that
\begin{equation*}
  \assert{\underline{I}_{\ell}}\
  \overline{\mathtt{x}} := \fun{f} (\overline{p})
  \ \assert{\underline{I}_{\ell'}}
  \hspace{0.5em}
  \text{or, equivalently,}
  \hspace{0.5em}
  \assert{\underline{I}_{\ell}}\ 
  \overline{\mathtt{u}}^\fun{f} := \overline{p};\
  \overline{\mathtt{r}}^\fun{f} := \fun{f} (\overline{\mathtt{u}}^\fun{f});\
  \overline{\mathtt{x}} := \overline{\mathtt{r}}^\fun{f}
  \ \assert{\underline{I}_{\ell'}}
\end{equation*}
for every $(\ell, \ell') \in \hat{E}$. 
By the inductiveness of $\Pi (\underline{G}^{\fun{main}}_k, \TT)$, we have
$\assert{\underline{I}_{\ell}}\ \overline{\mathtt{u}}^\fun{f} := \overline{p}\
\assert{\underline{I}_{s_k^\fun{f}}}$ and
$\assert{\underline{I}_{e_k^\fun{f}}}\ \overline{\mathtt{x}} :=
\overline{\mathtt{r}}^\fun{f}\ \assert{\underline{I}_{\ell'}}$.
Moreover,
$\assert{\underline{I}_{s_k^\fun{f}}}\
\overline{\mathtt{r}}^\fun{f} := \fun{f}
(\overline{\mathtt{u}}^\fun{f})\ \assert{\underline{I}_{e_k^\fun{f}}}$
by Proposition~\ref{propposition:strengthen_postcondition} and \ref{proposition:check_summary}.
Therefore
\begin{prooftree}
  \AxiomC{$\assert{\underline{I}_{\ell}}\
    \overline{\mathtt{u}}^\fun{f} := \overline{p}\
    \assert{\underline{I}_{s_k^\fun{f}}}$}

  \AxiomC{$\assert{\underline{I}_{s_k^\fun{f}}}\
    \overline{\mathtt{r}}^\fun{f} := \fun{f} (\overline{\mathtt{u}}^\fun{f})\
    \assert{\underline{I}_{e_k^\fun{f}}}$}

  \AxiomC{$\assert{\underline{I}_{e_k^\fun{f}}}\
    \overline{\mathtt{x}} := \overline{\mathtt{r}}^\fun{f}\
    \assert{\underline{I}_{\ell'}}$}

  \TrinaryInfC{$\assert{\underline{I}_{\ell}}\
    \overline{\mathtt{u}}^\fun{f} := \overline{p};\
    \overline{\mathtt{r}}^\fun{f} := \fun{f} (\overline{\mathtt{u}}^\fun{f});\
    \overline{\mathtt{x}} := \overline{\mathtt{r}}^\fun{f}\
    \assert{\underline{I}_{\ell'}}$}
  \UnaryInfC{$\assert{\underline{I}_{\ell}}\ 
    \overline{\mathtt{x}} := \fun{f} (\overline{p})\
    \assert{\underline{I}_{\ell'}}$}
\end{prooftree}
