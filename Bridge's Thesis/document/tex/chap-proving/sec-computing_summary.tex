
\section{Computing Summaries}\label{sec:computing-summary}
\todo[inline]{Rewrite marking function}
In order to help extract summaries from the inductive invariant,
MarkLocs$(\prog{P}_k)$ annotates in $\prog{P}_k$ the \emph{outermost} pair of
the entry and exit locations ${s_i}$ and ${e_i}$ of each unwound function call
to $\fun{f}$ with an additional superscript $\fun{f}$, i.e., $s_i^\fun{f}$ and
$e_i^\fun{f}$.
Outermost is defined by \emph{dominance} relation over program locations(nodes).
A location $\ell$ \emph{strictly dominates} $\ell'$~($\ell$ \emph{dom} $\ell'$)
if and only if every execution path from $s$ to $\ell'$ must go through $\ell$.
Similarly, $\ell$ \emph{post-dominates} $\ell'$~($\ell$ \emph{post-dom} $\ell'$)
if and only if every execution path from $\ell'$ to $e$ must go through $\ell$.
We can thereby define that $\langle s_i, e_i \rangle$ \emph{encloses}
$\langle s_j, e_j \rangle$ if only if $s_i$ \emph{dom} $s_j$ and $e_i$
\emph{post-dom} $e_j$,
and we can then find outermost pair with this definition that this pair is not
enclosed by any other pair through Algorithm~\ref{algorithm:mark-locations}.
Note that MarkLocs$(\prog{P}_k)$ is designed to handle unwound programs from
both unwinding methods,
and it does not provide any output but modify program $\prog{P}_k$ directly.

\todo[inline]{
The implementation marks different locations than those described above.
Here we say it marks the entry and exit locations of unwound body.
However, we actually use the locations before and after the function call.
That is, the location before the formal parameters are assigned and the location
after the receiving variable receives value from return variable.
This is because the invariants may contain invariants with less variable to be
quantified~(Actually this is observed from \textsc{CPAchecker}).
}

\begin{algorithm}
\begin{doublespace}
  \KwIn{$\prog{{P}}_k$: a program}
  \If{$\prog{{P}}_k$ is from $\method{unwind}$}
  {
    \ForEach{$s_i, e_i \in V$ of ${G}^\fun{main}$}
    {
      $\fun{f}$ := $\method{getFunction}(i)$;

      \If{$\neg(\exists j. \fun{f} = \method{getFunction}(j)
           \land s_j$ dom $s_i \land e_j$ post-dom $e_i)$}
      {
        Change $s_i, e_i$ to $s^\fun{f}_i, e^\fun{f}_i$;
      }
    }
    \Return;
  }
  \If{$\prog{{P}}_k$ is from $\method{unwind}_\prog{P}$}
  {
    \ForEach{function $G^{\fun{f}_i}$ in the program $\prog{{P}}_k$}
    {
      \If{$\neg(\exists j < i \land s^\fun{f}_j, e^\fun{f}_j \in V$ of $G^{\fun{f}_j})$}
      {
        Change $s_i, e_i$ to $s^\fun{f}_i, e^\fun{f}_i$;
      }
    }
    \Return;
  }
\end{doublespace}
  \caption{
  $\textmd{MarkLocs}(\prog{P}_k)$}
  \label{algorithm:mark-locations}
\end{algorithm}

Now, let the CFG for the $\fun{main}$ function $\underline{G}^{\fun{main}}_k =
\langle V, E, \underline{\cmd{cmd}}^{\fun{main}},
\overline{\mathtt{u}}^{\fun{main}}, \overline{\mathtt{r}}^{\fun{main}},s,e
\rangle$.
Function ComputeSummary($\prog{P}_k$, $\Pi (\underline{G}^{\fun{main}}_k, \TT)$)
extracts summaries from the inductive invariant
$\Pi (\underline{G}^{\fun{main}}_k, \TT)= \{ I_\ell : \ell \in V\}$
(Algorithm~\ref{algorithm:compute-summary}).

\begin{algorithm}
\begin{doublespace}
  \KwIn
  {
    $\prog{P}_k$: a program;
    $\{ I_\ell : \ell \in V \}$: an inductive invariant of $\underline{G}^{\fun{main}}_k$
  }
  \KwOut{$S[\bullet]$: function summaries}

  MarkLocs($\prog{P}_k$);

  \ForEach{function $\fun{f}$ in the program $\prog{P}_k$}
  { 
    $S[\fun{f}]$ := $\TT$\;
    \ForEach{pair of locations $\langle s_i^\fun{f},e_i^\fun{f} \rangle \in V\times V$}
    {
      \If{$I_{s_i^\fun{f}}$ contains return variables of $\fun{f}$}
      {
        $S[\fun{f}]$ := $S[\fun{f}]\wedge \forall X_\fun{f}. I_{e_i^\fun{f}}$
      }
      \Else
      {
        $S[\fun{f}]$ := $S[\fun{f}]\wedge \forall X_\fun{f}. (I_{s_i^\fun{f}} \implies I_{e_i^\fun{f}})$
      }
    }
    
  }

  \Return $S[\bullet]$\;
\end{doublespace}
  \caption{
  $\textmd{ComputeSummary}(\prog{P}_k, \Pi (\underline{G}^{\fun{main}}_k, \TT))$}
  \label{algorithm:compute-summary}
\end{algorithm}

For each function $\fun{f}$ in the program $\prog{P}_k$,
we first initialize its summary $S[\fun{f}]$ to $\TT$.
The set $X_\fun{f}$ contains all variables appearing in
$\underline{G}^{\fun{main}}_k$ except the set of formal parameters and return
variables of $\fun{f}$.
For each pair of locations $(s_i^\fun{f},e_i^\fun{f})\in V\times V$ in $\underline{G}^{\fun{main}}_k$,
if the invariant of location $s_i^\fun{f}$ contains return variables of $\fun{f}$,
we update $S[\fun{f}]$ to the formula $S[\fun{f}]\wedge \forall X_\fun{f}.I_{e_i^\fun{f}}$.
Otherwise, we update it to a less restricted version $S[\fun{f}]\wedge \forall X_\fun{f}. (I_{s_i^\fun{f}} \implies I_{e_i^\fun{f}})$ (Figure~\ref{figure:updating-summary}).

\begin{figure}[t]
  \centering
    \begin{tikzpicture}[scale=1.2,->,>=stealth',shorten >=1pt,auto,node
      distance=2cm,thick,node/.style={circle,draw,minimum width=0.8cm,inner sep=0}]

      \draw [fill=gray!10] (3.5, -1) ellipse (1.8 and 1.5);
      \node (text) at (3.5, -1) {$\method{rename}(G_\fun{f},i)$};
      \node[node] (10) at (3.5, 0)  {$s_i^\fun{f}$};
      \node[node] (11) at (3.5, -2) {$e_i^\fun{f}$};

      \node (arrow_s) at (5.5, -1) {};
      \node (arrow_e) at (6.5, -1) {};
      
      \node[rectangle,text centered,draw] (text) at (9, -1)
      { add $\forall X_{\fun{f}}.(I_{s_i^\fun{f}} \implies I_{e_i^\fun{f}})$
        to $S[\fun{f}]$}; 
      
      \path
        (arrow_s) edge [dotted]
                  node {} (arrow_e);
    \end{tikzpicture}

  \caption{Updating a Summary}
  \label{figure:updating-summary}
\end{figure}
\begin{proposition}
\label{propposition:invariant}
Let $Q$ be a formula over all variables in $\underline{G}^{\fun{main}}_k$ except $\overline{\mathtt{r}}^\fun{f}$.
We have $\assert{Q}\
  \overline{\mathtt{r}}^\fun{f} := \fun{f}
  (\overline{\mathtt{u}}^\fun{f})\ \assert{Q}$.
\end{proposition}
The proposition holds because the only possible overlap of variables in $Q$ and
in $\overline{\mathtt{r}}^\fun{f} := \fun{f} (\overline{\mathtt{u}}^\fun{f})$
are the formal parameters $\overline{\mathtt{u}}^\fun{f}$.
However, we assume that values of formal parameters do not change in a function
(see Section~\ref{ch:preliminaries});
hence the values of all variables in $Q$ stay the same after the execution of
the function call $\overline{\mathtt{r}}^\fun{f} := \fun{f} (\overline{\mathtt{u}}^\fun{f})$.

\begin{proposition}
  \label{propposition:strengthen_postcondition}
  Given CFG $\underline{G}^{\fun{main}}_k = \langle V, E,
    \underline{\cmd{cmd}}^{\fun{main}}, \overline{\mathtt{u}}^{\fun{main}},
    \overline{\mathtt{r}}^{\fun{main}},s,e \rangle$.
  If $\assert{\TT}\ \overline{\mathtt{r}}^\fun{f} := \fun{f}
     (\overline{\mathtt{u}}^\fun{f})\ \assert{S[\fun{f}]}$ holds, then
  $\assert{I_{s_i^\fun{f}}}\ \overline{\mathtt{r}}^\fun{f} := \fun{f}
   (\overline{\mathtt{u}}^\fun{f})\ \assert{
     I_{e_i^\fun{f}}}$ for all $(s_i^\fun{f},e_i^\fun{f})\in V\times V$.
\end{proposition}
For each pair $(s_i^\fun{f},e_i^\fun{f})\in V\times V$, we consider two cases:
\begin{enumerate}
\item $I_{s_i^\fun{f}}$ contains some return variables of $\fun{f}$:\\
In this case, the conjunct $\forall X_\fun{f}.I_{e_i^\fun{f}}$ is a part of $S[\fun{f}]$, we then have
\begin{prooftree}
  \AxiomC{$\assert{\TT}\ \overline{\mathtt{r}}^\fun{f} := \fun{f}
       (\overline{\mathtt{u}}^\fun{f})\ \assert{S[\fun{f}]}$ }
  \RightLabel{Postcondition Weakening}
    \UnaryInfC{$\assert{\TT}\ \overline{\mathtt{r}}^\fun{f}:=\fun{f}(\overline{\mathtt{u}}^\fun{f})\ \assert{\forall X_\fun{f}. I_{e_i^\fun{f}}}$}
  \RightLabel{Postcondition Weakening}
  \UnaryInfC{$\assert{\TT}\ \overline{\mathtt{r}}^\fun{f} := \fun{f}(\overline{\mathtt{u}}^\fun{f})\ \assert{I_{e_i^\fun{f}}}$}
  \RightLabel{Precondition Strengthening}
    \UnaryInfC{$\assert{I_{s_i^\fun{f}}}\ \overline{\mathtt{r}}^\fun{f} := \fun{f}(\overline{\mathtt{u}}^\fun{f})\ \assert{I_{e_i^\fun{f}}}$}
\end{prooftree}

\item $I_{s_i^\fun{f}}$ does not contain any return variables of $\fun{f}$:\\
In this case, the conjunct $\forall X_\fun{f}.(I_{s_i^\fun{f}} \implies I_{e_i^\fun{f}})$ is a part of $S[\fun{f}]$, we then have
\begin{prooftree}
 \AxiomC{}
 \LeftLabel{Prop.~\ref{propposition:invariant}}
 \UnaryInfC{$\assert{I_{s_k^\fun{f}}}\ 
    \overline{\mathtt{r}}^\fun{f} := \fun{f} (\overline{\mathtt{u}}^\fun{f})\
    \assert{I_{s_k^\fun{f}}}$}
\AxiomC{$\assert{\TT}\ \overline{\mathtt{r}}^\fun{f} := \fun{f}
       (\overline{\mathtt{u}}^\fun{f})\ \assert{S[\fun{f}]}$ }
  %\RightLabel{Weakening
\UnaryInfC{$\assert{\TT}\ \overline{\mathtt{r}}^\fun{f}:=\fun{f}(\overline{\mathtt{u}}^\fun{f})\ \assert{\forall X_\fun{f}. (I_{s_k^\fun{f}} \implies I_{e_k^\fun{f}})}$}
%\RightLabel{Weakening}
\UnaryInfC{$\assert{\TT}\ \overline{\mathtt{r}}^\fun{f}:=\fun{f}(\overline{\mathtt{u}}^\fun{f})\ \assert{I_{s_k^\fun{f}} \implies I_{e_k^\fun{f}}}$}
%\RightLabel{Strengthening}
 \UnaryInfC{$\assert{I_{s_k^\fun{f}}}\
    \overline{\mathtt{r}}^\fun{f} := \fun{f} (\overline{\mathtt{u}}^\fun{f})\
    \assert{I_{s_k^\fun{f}} \implies I_{e_k^\fun{f}}}$}

 \BinaryInfC{$\assert{I_{s_k^\fun{f}}}\ 
    \overline{\mathtt{r}}^\fun{f} := \fun{f} (\overline{\mathtt{u}}^\fun{f})\
    \assert{I_{e_k^\fun{f}}}$}
\end{prooftree}
\end{enumerate}

