
\section{Unwinding}\label{sec:unwinding}
As introduced in Chapter~\ref{ch:overview}, our algorithm produces more and more
accurate under-approximations through unwinding the program.
An important feature of unwinding is that it should keep the behavior of unwound
program being the same as the original one.
In this section, we introduce two function unwinding methods accompanied with
simple proofs of the equivalence between the original and unwound functions.
We also devise different strategies to produce unwound programs when using
different function unwinding methods.

\subsection{Function Inlining Method}\label{subsec:inlining}
The function inlining method directly follows the concept of inline function in
C/C++ language and compiler optimizations.
It expands the body at the location of a function call.
We can achieve this in our program model.
The function $\method{unwind}(G^\fun{f})$ returns a CFG $K^\fun{f}$
obtained by replacing all function call edges in $G^\fun{f}$ with the CFGs of
the called functions~(Figure~\ref{figure:unwinding}).
The formal definition is given below.

\begin{figure}[t]
  \centering
    \begin{tikzpicture}[scale=1.2,->,>=stealth',shorten >=1pt,auto,node
      distance=2cm,thick,node/.style={circle,draw,minimum width=0.8cm,inner sep=0}]

      \node[node] (0) at (-4, 0)  {$\ell$};
      \node[node] (1) at (-4, -2) {$\ell'$};

      \draw [fill=gray!10] (4, -1) ellipse (1.8 and 1.5);
      \node (text) at (4, -1) {$\method{rename}(G_\fun{g},i)$};
      \node[node] (00) at (0, 0)  {$\ell$};
      \node[node] (01) at (0, -2) {$\ell'$};
      \node[node] (10) at (4, 0)  {\smallnode{$s_i$}};
      \node[node] (11) at (4, -2) {\smallnode{$e_i$}};
      \node (arrow_s) at (-2.5, -1) {};
      \node (arrow_e) at (-0.5, -1) {};

      \path
        (arrow_s) edge [dotted]
                  node {} (arrow_e)
        (0) edge
            node {$\overline{\mathtt{x}} := \fun{g} (\overline{p})$} (1)

        (00) edge
             node {$\overline{\mathtt{u}}_i^\fun{g} := \overline{p}$} (10)

        (11) edge
             node {$\overline{\mathtt{x}} := \overline{\mathtt{r}}_i^\fun{g}$} (01) ;
    \end{tikzpicture}

  \caption{Unwinding Function Calls}
  \label{figure:unwinding}
\end{figure}

We first define the function $\method{rename}(G^\fun{f},i)$.
It returns a CFG obtained by first replacing every return command
$\mathtt{return}\ \overline{q}$ with assignments to return variables
$\overline{\texttt{r}}^\fun{f} := \overline{q}$ and then renaming all variables
and locations in $G^\fun{f}$ with the given index value $i$.
The function $\method{addindex}(c, i)$ is used to rename all variables in a
command $c$ with index $i$.
Formally,
\[
\begin{array}{l}
\method{rename}(\langle V, E,\cmd{cmd}^\fun{f},\overline{\texttt{u}}^\fun{f},
\overline{\texttt{r}}^\fun{f},s,e \rangle,i)
= \langle V_i, E_i,\cmd{cmd}_i^\fun{f},\overline{\texttt{u}}_i^\fun{f},
\overline{\texttt{r}}_i^\fun{f},s_i,e_i \rangle \text{ such that}\\

V_i = \{\ell_i:\ell \in V\} \\

E_i = \{(\ell_i,\ell'_i): (\ell,\ell') \in E\} \\

\cmd{cmd}_i^\fun{f}(\ell_i,\ell'_i) =
  \left\{
  \begin{array}{ll}
   \method{addindex}(\overline{\texttt{r}}^\fun{f}:=\overline{q}, i) &
     \text{if }\cmd{cmd}^\fun{f}(\ell,\ell')=\mathtt{return}\ \overline{q} \\
   \method{addindex}(\cmd{cmd}^\fun{f}(\ell,\ell'), i) &
     \text{otherwise}
  \end{array}
  \right.
\end{array}
\]
Then, given a CFG $G^\fun{f}=\langle V, E,\cmd{cmd}^\fun{f},
\overline{\texttt{u}}^\fun{f}, \overline{\texttt{r}}^\fun{f},s,e \rangle$,
we use $\hat{E} =\{(\ell, \ell')\in E: \cmd{cmd}^\fun{f} (\ell,\ell')=(\overline{\texttt{x}}:=\fun{g}(\overline{p}))\}$ to denote the set of function call edges in $E$ and define a function $\method{idx}(e)$ that maps a call edge $e$ to a unique index.
\hide{ % TODO
The predicate below then can be used to represent a renamed CFG $\method{rename}(G^\fun{g},i)$ according to a function call edge $(\ell, \ell')$ in $G^\fun{f}$.
\[
\begin{split}
{UnwindCall}(\ell,\ell') \equiv (\ell,\ell')\in\hat{E} \wedge \cmd{cmd}^\fun{f}(\ell,\ell')=(\overline{\texttt{x}}:=\fun{g}(\overline{p}))\wedge\method{idx}(\ell, \ell')=i  \\
  \wedge \method{rename}(G^\fun{g},i)=\langle V_i, E_i,\cmd{cmd}^\fun{g}_i, \overline{\texttt{u}}^\fun{g}_i, \overline{\texttt{r}}^\fun{g}_i,s_i,e_i\rangle
\end{split}
\]
}
Finally, we provide the definition of $\method{unwind}(G^\fun{f})$.
\[
\begin{array}{l}
\method{unwind}(G^\fun{f}) = \langle V_u, E_u,\cmd{cmd}_u^\fun{f}, \overline{\texttt{u}}^\fun{f}, \overline{\texttt{r}}^\fun{f},s,e \rangle \text{ where} \\
\end{array}
\]
\[
\begin{array}{l}
\begin{aligned}
 V_u = V\cup\bigcup\{V_i:(\ell,\ell')\in\hat{E} \wedge \cmd{cmd}^\fun{f}(\ell,\ell')=(\overline{\texttt{x}}:=\fun{g}(\overline{p}))\wedge\method{idx}(\ell, \ell')=i  \\
  \wedge \method{rename}(G^\fun{g},i)=\langle V_i, E_i,\cmd{cmd}^\fun{g}_i, \overline{\texttt{u}}^\fun{g}_i, \overline{\texttt{r}}^\fun{g}_i,s_i,e_i\rangle \}
\end{aligned}\\

\begin{aligned}
 E_u = E \setminus \hat{E} \cup \bigcup\{E_i\cup\{(\ell, s_i),(e_i, \ell')\}: (\ell, \ell')\in \hat{E} \wedge \cmd{cmd}^\fun{f}(\ell, \ell')=(\overline{\texttt{x}}:=\fun{g}(\overline{p})) \\
  \wedge \method{idx}(\ell, \ell')=i \wedge \method{rename}(G^\fun{g},i) =
\langle V_i, E_i,\cmd{cmd}^\fun{g}_i, \overline{\texttt{u}}^\fun{g}_i, \overline{\texttt{r}}^\fun{g}_i,s_i,e_i\rangle \}
\end{aligned}\\

\cmd{cmd}_u^\fun{f} (\ell, \ell') =
\left\{
  \begin{array}{ll}
  (\overline{\texttt{u}}^\fun{g}_i:=\overline{p}) &\text{if } (\ell,\ell')=(\ell,s_i)
  \\
  (\overline{\texttt{x}}:=\overline{\texttt{r}}^\fun{g}_i) &\text{if } (\ell,\ell')=(e_i,\ell') \\
  \cmd{cmd}^\fun{g}_i(\ell, \ell') &\text{if } (\ell, \ell') \in E_i \\
  \cmd{cmd}^\fun{f}(\ell,\ell') &\text{otherwise}
  \end{array}
\right.
\end{array}
\]

\begin{proposition}
  Let $G^\fun{f}$ be a control flow graph. $P$ and $Q$ are logic formulas with
  free variables over program variables of $G^\fun{f}$. $\assert{P}\ G^\fun{f}\
  \assert{Q}$ if and only if $\assert{P}\ \method{unwind}(G^\fun{f})\ \assert{Q}$.
\end{proposition}
\todo[inline]{Describe why unwound function and original function have the same
 behavior via simple Hoare rule}
Essentially, $G^\fun{f}$ and $\method{unwind}(G^\fun{f})$ represent the same
function $\fun{f}$. The only difference is that the latter has more program
variables after unwinding,
but this does not affect the states over program variables of $G^\fun{f}$
before and after the function.

\subsubsection*{Unwinding Program}
Considering the \method{unwind}, it transforms a single CFG into another
single CFG with the same behavior but more program variables.
Following our program, the program starts at $\fun{main}$ function.
Therefore, to unwind the program in each iteration of 
Algorithm~\ref{algorithm:overview},
we can simply apply \method{unwind} on $G^\fun{main}$ function repeatedly.
In other words,
\[
\begin{array}{l}
\text{Initially, } \prog{P}_0 = \prog{P}
    \therefore G^\fun{main}_0 = G^\fun{main} \\
G^\fun{main}_k = \method{unwind}(G^\fun{main}_{k-1}) \\
\prog{P}_k = \prog{P}_{k-1} 
             \setminus \{G^\fun{main}_{k-1}\} \cup \{G^\fun{main}_k\}
\end{array}
\]


\subsection{Function Duplicating Method}\label{subsec:duplicating}
The previously mentioned method comes with an obvious drawback that it will
cause exponential increase in size of the unwound program.
Considering the McCarthy 91 function, the number of recursive calls in the
program doubles in the unwound function at each iteration of
Algorithm~\ref{algorithm:overview},
and the program size doubles at next iteration since every function call is
inlined, i.e., transformed to a copy of the function body.
Therefore, we devised another method $\method{unwind}_\prog{P}(G^\fun{f}, k)$
that duplicates functions in program \prog{P} instead of inlining function calls
to prevent exponential growth.
Note that the $\method{unwind}_\prog{P}(G^\fun{f}, k)$ also preserves the 
behavior of original function $\fun{f}$ in the modified program \prog{P} after
unwinding.

The procedure of the new method is simple.
We give a function body $G^\fun{f}$ which contains multiple function call edges
to the same functions $\fun{g}$ as an example.
In the old method, each function call to $\fun{g}$ is replaced by almost the
same function body $G^\fun{g}$ except being indexed during unwinding,
and therefore the program size grows exponentially.
The new method creates a new function $\fun{g}_k$ with a given index value $k$.
The function $\fun{g}_k$ has exactly the same function body $G^\fun{g}$.
Those calls to $\fun{g}$ are then replaced with calls to $\fun{g}_k$.
Only one additional function body is introduced for multiple function
calls.
Notice that, since there are a modified version of $G^\fun{f}$ and many 
duplicated functions $G^{\fun{g}_k}$, $\method{unwind}_\prog{P}$ returns a pair 
of a function and a set of functions instead of one function.
The formal definition is given as follows.

\missingfigure{Demonstrate alternative unwinding}

First, we define $\method{duplicate}_\prog{P}(\fun{g}, k)$ to copy $\fun{g}$ in
the program \prog{P}.
Given $k$ indicating the $k$-th iteration in Algorithm~\ref{algorithm:overview},
we create a new function $\fun{g}_k$ with the body $G^{\fun{g}_k}$ which is
almost the same as $G^\fun{g}$ except being indexed via \method{rename}.
\[
  \method{duplicate}_\prog{P}(G^\fun{g}, k) = G^{\fun{g}_k} \text{ where }
  \fun{g}_k \not\in \prog{P} \land G^{\fun{g}_k}=\method{rename}(G^\fun{g}, k)
\]
Then, to unwind a given CFG $G^\fun{f}$, we produce copies of the functions
called by $\fun{f}$ and replace all call edge by calls to the duplication.
Formally, the definition of $\method{unwind}_\prog{P}(G^\fun{f})$ is as below.
~(Please refer to \S~\ref{subsec:inlining} for definitions of the symbols).
\[
\begin{array}{l}
\method{unwind}_\prog{P}(G^\fun{f}, k)
  = \langle G^\fun{f}_k, DUP^\fun{f}_k) \rangle \text{ where} \\
DUP^\fun{f}_k = \{G^{\fun{g}_k} :
    \cmd{cmd}^\fun{f}(\ell,\ell')=(\overline{\texttt{x}}:=\fun{g}(\overline{p}))
    \land G^{\fun{g}_k} = \method{duplicate}_\prog{P}(G^\fun{g}, k)
  \}\\
G^\fun{f}_k = \langle V, E,\cmd{cmd}_u^\fun{f}, \overline{\texttt{u}}^\fun{f},
      \overline{\texttt{r}}^\fun{f},s,e \rangle \\
{%\multicolumn{2}{l}{
\cmd{cmd}_u^\fun{f} (\ell, \ell') =
\left\{
  \begin{array}{ll}
  (\overline{\texttt{x}}:=\fun{g}_k(\overline{p})) &\text{if } 
    \cmd{cmd}^\fun{f}(\ell,\ell')=(\overline{\texttt{x}}:=\fun{g}(\overline{p})) \\[-8pt]
  & \land\ G^{\fun{g}_k} = \method{duplicate}_\prog{P}(G^\fun{g}, k)\\
  \cmd{cmd}^\fun{f}(\ell,\ell') &\text{otherwise}
  \end{array}
\right.
}
\end{array}
\]
Notice that, if function $\fun{f}$ contains multiple calls to the same function
$\fun{g}$,
the CFG $G^{\fun{g}_k}$ is not introduced multiple times by set union.
This is the primary reason that leads to less increase in program size.

\begin{proposition}\label{prop:unwinding-duplicate}
  Let $\fun{f}$ be a function without self or mutual recursion and 
  $G^\fun{f}$ be its control flow graph.
  $P$ and $Q$ are logic formulas with free variables over program variables of
  $G^\fun{f}$.
  $\assert{P}\ G^\fun{f}\ \assert{Q}$ if and only if 
  $\assert{P}\ \method{unwind}_\prog{P}(G^\fun{f}, k)\ \assert{Q}$.
\end{proposition}
In this method, the only difference between $G^\fun{f}$ and
$\method{unwind}_\prog{P}(G^\fun{f}, k)$ is that each function call to
$\fun{g}$ is substituted with call to $\fun{g}_k$.
Hence, the proposition should trivially hold if the behavior of $\fun{g}$ and
$\fun{g}_k$ have the same behavior.
Under the premise that $\fun{f}$ is a function without recursion,
$\fun{g}_k$ won't introduce new recursion since it does not belongs to original
program \prog{P}.
Therefore, $\fun{g}_k$ and $\fun{g}$ simply behave the same because 
$G^{\fun{g}_k} = \method{rename}(G^\fun{g},k)$.
Without the premise, $\fun{g}$ can directly or indirectly call $\fun{f}$,
and introducing $\fun{g}_k$ complicates the mutual recursion further.
Thus, proving the proposition without the premise is much harder.
Since the stronger proposition is not necessary for our work,
we don't provide further proof here.

\subsubsection*{Unwinding Program}
Unwinding the program with $\method{unwind}_\prog{P}$ is very different from
the one with $\method{unwind}$ because we have to add duplication of functions
into the program \prog{P}.
Repeatedly applying $\method{unwind}_\prog{P}$ on $\fun{main}$ does not help
this time since it only copies the functions called by $\fun{main}$ again
and again.
In our approach, we apply unwinding on $\fun{main}$ in the first iteration,
and on only those duplications of functions in following iterations.
This is crucial because the $\fun{main}$ function and the duplications are
guaranteed to be not involved in recursive calls and fulfill the premise in
Proposition~\ref{prop:unwinding-duplicate}.

Since there are multiple functions being unwound in an iteration,
we extend $\method{unwind}_\prog{P}$ to handle a set of functions by merging
all results into a set.
The extended method therefore returns a pair of sets of functions.
Formally, given a set of CFGs $S$,
\[
\begin{array}{l}
\method{unwind}_\prog{P}(S, k) = \langle GS_k, DUPS_k \rangle \text{ where}\\
GS_k = \{G^\fun{f}_k : G^\fun{f} \in S
  \land \method{unwind}_\prog{P}(G^\fun{f}, k) 
        = \langle G^\fun{f}_k, DUP^\fun{f}_k \rangle
  \} \\
  
DUPS_k = \bigcup\{
  DUP^\fun{f}_k : G^\fun{f} \in S
  \land \method{unwind}_\prog{P}(G^\fun{f}, k) 
        = \langle G^\fun{f}_k, DUP^\fun{f}_k \rangle
  \} \\
\end{array}
\]

Then, following the simpler representation, we unwind the program \prog{P}
via applying $\method{unwind}_\prog{P}$ on duplications $DUPS$.
That is,
\[
\begin{array}{l}
\text{Initially, } \prog{P}_0 = \prog{P}, GS_0 = \emptyset,DUPS_0 = \{G^\fun{main}\} \\
\langle GS_k, DUPS_k \rangle = \method{unwind}_\prog{P}(DUPS_{k-1}, k)\\
\prog{P}_k = (\prog{P}_{k-1} \setminus GS_{k-1}) \cup GS_k \cup DUPS_k\\
\end{array}
\]
As defined above, $DUPS_k$ represents all those duplications of functions in
$k$-th iteration.
In $k$-th iteration, we only unwind functions in $DUPS_{k-1}$ because they are
new functions and must not involve in any recursion.
As a result, the unwound program $\prog{P}_k$ could preserve the behavior of
original program \prog{P}.


\todo[inline]{
Move following words to a better section

\{
This method requires the underlying analyzers dealing with
recursion-free functions.
Fortunately most modern analyzers support this feature.
\}
\{
It still provides more and more accurate under-approximation in each
iteration.
\}
}
