\newcommand{\eng}[1]{\ \raisebox{1pt}{#1}}

\chapter{摘要}

在本篇論文中,我們介紹並實作了兩個針對線性程式實施軟體測試與模型合成的方法,分別基於學習與抽樣。利用這兩個全新的方式,我們希望能夠在提供一個量化的保證與合理的資源耗用下,減低軟體測試與正規驗證之間的分別。\newline

本篇論文中,基於學習機制的方式的進行,是從學習一個程式所有可行的執行軌跡之集合開始,得出一模擬此集合的模型,再利用測試來觀察此模型是否包含了錯誤的行為。準確的學習演算法需要保證模型與程式的完全相似,而此問題基本上是無法決定的。因此,我們的學習機制利用了\eng{Probably Approximately Correct (PAC)}學習架構,其特色為應用抽樣來解決模型與程式是否相同的問題,並且利用名為錯誤率\eng{(error probability)}與信心水平\eng{(confidence level)}的參數來提供正確性的保證。除此之外,我們的基於學習機制之方式也會將模擬程式行為的模型輸出,此模型也保有上述之正確性保證。\newline

我們並提出了另外一個基於取樣來實作的方式。此法也是根據PAC學習機制來提供正確性的保證。我們利用了具象符號測試\eng{(Concolic Testing)}工具來作為我們的取樣者,並從程式中抽取特定數量的執行軌跡。最終,再從已抽取之執行軌跡的集合中分析,來提供關於正確性的保證,從中找到錯誤執行軌跡,或是因為計算資源的不足而無法完成分析。\newline

我們實作了一個稱為\PACMAN 的工具,此工具提供兩種運行模式,分別實作了上述所提及之兩種機制。除此之外,我們在實驗中所獲得的初步結果中,\PACMAN 的表現相當優異,在某些範例上甚至超越了一些已經成熟的軟體驗證工具。因此,我們將\PACMAN 提交至2016年軟體驗證競賽(Software Verification Competition)。在競賽中的遞迴\eng{(Recursive)}項目中我們獲得了第五名,而在陣列可到達性\eng{(Array-reach)}項項目中我們則是得到了第四名。另外,我們將一篇描述了基於學習機制的方式之論文提交至2016年國際軟體工程會議(International Conference on Software Engineering),並且成功地被採用。\newline

關鍵字:軟體測試,程式分析,模型合成,\eng{PAC}學習


%在程序分析 \eng{(Program Analysis)} 領域中,
%分析遞迴程式 \eng{(Recursive Program)} 是一項難以處理的課題。
%現有程式分析工具往往因為無法處理程式中的遞迴函式,
%既而忽略部分程式碼不進行分析,或甚至拒絕檢驗遞迴程式。
%本篇論文針對遞迴程式的分析提出新的觀點與分析方法;
%作者認為與其重新設計演算法且實作新的分析器,
%不如改進現有的程式分析工具。
%但更改他人實作的軟體工具並非易事,
%因此本文提出的方法將現有的工具視為黑箱不做任何更動,
%反而是利用程式變換方法 \eng{(Program Transformation)} 將待分析的遞迴
%程式轉換成無遞迴的程式,再交給黑箱分析工具進行檢驗;
%並使用黑箱工具提供被分析程式中的不變量 \eng{(Invariants)},
%進而證明原遞迴程式的正確性。
%
%本篇作者與實驗室團隊受惠於程式分析工具 \eng{\textsc{CPAchecker}},
%實作出此演算法的雛型 \eng{\textsc{CPArec}},
%且參與 \eng{2015} 年度軟體驗證競賽 \eng{(Competition on Software Verification)};
%和其它頂尖實驗室所開發工具相較,我們的工具針對遞迴程式進行分析時,
%亦表現出相當的效率及效能,獲得第三名的佳績。
%
%關鍵字:軟體驗證、程序分析、靜態分析、遞迴程式、程式變換方法
