\chapter{The Sampling-based Procedure}\label{ch:sampling_proc}

In this chapter, we propose another approach to provide guarantees to a program's correctness, which we will refer to as the sampling-based procedure. The objective of this procedure is the same as the learning-based procedure described in Chapter \ref{ch:learning_proc}, except that when we prove a program $\PAC$, we do not additionally provide an approximation to the target program. 

The idea behind the sampling-based procedure is straightforward. As stated in Section \ref{sec:pac_learning}, consider hypothesis \ref{eq:pac_hypothesis}:
\begin{equation*}
Prob_{w \in \mathcal{U}|D} [\neg \varphi (w)] \leq \epsilon
\end{equation*}
In the setting of the sampling-based procedure, the universe is $\Sigma^\ast$, and the language of the target program is $R \subseteq \Sigma^\ast$. However, the task of the sampling-based procedrue is to check whether the formula $\varphi(w) = w \notin R \cap L(B)$ is $PAC(\epsilon, \delta)-$valid, i.e. whether $Prob_{w \in \Sigma^\ast | D } [w \in R \cap L(B)] \leq \epsilon$ with confidence of at least $\delta$. Simply put, we want to provide a statistical guarantee about program correctness when sampling does not reveal any error in the program-under-test. In order to provide such a guarantee, we need to modify the contents of Chapter \ref{ch:PAC} and Chapter \ref{ch:eq} so that they can fit the sampling-based scenario.

In this technique, we also utilize the concept of batched samples described in Section \ref{sec:apply_concolic}. With this in mind, the universe in the previously mentioned guarantee needs to be changed to $\mathcal{U}_k = (\Sigma^\ast)^k$ with respect to the distribution $D_k$, where $k$ is the batch size and $D_k$ is the distribution defined in Section \ref{sec:apply_concolic}. Also, recall the set of feasible decision vectors of a program $\DEC(\Pi)$ is in fact $R$. Then, our guarantee becomes:
\begin{equation}\label{eq:sampling_guarantee}
Prob_{S \in \mathcal{U}_k | D_k} [\exists w \in S : w \in \DEC(\Pi) \cap L(B)] \leq \epsilon
\end{equation} 
with confidence $\delta$. This guarantee is similar to the one mentioned in \ref{eq:equ_pac}. 

Recall $q_i$, the number of samples needed to establish that the target program is $\PAC$ given by Angluin in \cite{Angluin87}. When we perform the sampling-based procedure, we use a concolic tester to acquire $q_i$ batched samples. Then, for each batched sample $S$, we check whether there exists a $w$ such that $w \in S$ and $w \in L(B)$. Note that by definition $w \in \DEC(\Pi)$. If for all the batched samples no such $w$ is found, then based on the previously mentioned guarantee and \cite{Angluin87}, we can deduce that the program-under-test is $\PAC$, and terminate the procedure. On the other hand, if there really is such a $w$ that actually is an error feasible decision vector, then this $w$ is actually a witness to a bug within the program-under-test and reported to the user. The following lemma then shows that if we use $q_i$ batched samples for testing the correctness, we can obtain the guarantee in guarantee \ref{eq:sampling_guarantee}.

\begin{lemma}
Let $\epsilon$ and $\delta$ be the error and confidence parameters, and $R$ be the language of the target program. If no decision vector $w \in L(B)$ is found in the $q_i$ batched samples, then it holds that the program is $\PAC$.
\end{lemma}

However, it is not always the case that the program-under-test can be proved correct or containing an error feasible decision vector. In reality, proving the correctness of a program usually is done within a limited period of time. If the sampling-based mechanism cannot obtain $q_i$ samples within the time limit, then the procedure cannot by any means conclude whether the program is in fact correct or not. If this is the case, then our procedure reports unknown, indicating an incomplete session. 